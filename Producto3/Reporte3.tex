\documentclass{article}
\usepackage[export]{adjustbox}
\usepackage{graphicx}
\usepackage{fancyvrb}


\title{Reporte Actividad 3}
\author{Antonio Cota Rodr\'iguez}
\date{}


\begin{document}

\maketitle

\graphicspath{ {Imagenes/} }


\section*{Introducci\'on}

Con la presente pr\'actica su objetivo es introducirnos al lenguaje de programaci\'on Fortran. El siguiente reporte explica las actividades realizadas con este prop\'osito, se describe brevemente cada una, se a\~nade una imagen de la ejecuci\'on y se muestra su respectivo c\'odigo.
\section*{Actividades}

\subsection*{Area}
Con el siguiente programa podremos calcular el \'area de un c\'irculo dado su radio.\\ \\
\adjincludegraphics[scale=0.67,clip]{Area}
	
\begin{Verbatim}[frame=single]

! Area.f90 : Calculates the area of the circle, sample program
!--------------------------------------------------------------

Program Circle_area ! Begin main program

IMPLICIT NONE ! Declare all variables

Real *8 :: radius, circum, area ! Declare reals
Real *8 :: PI = 4.0 * atan(1.0) ! Declare , assign Real
Integer :: model_n = 1 ! Declare , assign Ints
print * , "Enter a radius:" ! Talk to user
read * , radius ! Read into radius
circum = 2.0 * PI * radius ! Calc circumference 
area = radius * radius * PI ! Calc area 
print * , "Program number = " , model_n ! Print program number
print * , "Radius = " , radius ! Print radius
print * , "Circumference = " , circum ! Print circumference
print * , "Area = " , area ! Print Area

End Program Circle_area ! End main program code


\end{Verbatim}

\subsection*{Volumen}
Calcular el volumen de un l\'iquido en un tanque esf\'erico cuando el l\'iquido est\'a a una altura h del fondo del tanque. En la ejecuci\'on se define el radio y la altura h.\\ \\
\adjincludegraphics[scale=0.67,clip]{Volumen_esfera}


\begin{Verbatim}[frame=single]

! Objetivo del programa:
! Obtener el volumen de una esfera dado el radio por el usuario

PROGRAM Volumen_Esfera

IMPLICIT NONE 

! Declaracion de variables

REAL :: Volumen
REAL :: radio
REAL, PARAMETER :: PI = 3.1416

!-----------------------------------------------------------------

! Obtener el valor del radio

Write (*,*) " Ingrese el radio de la esfera "
READ (*,*) radio

!-------------------------------------------------------------------

! Calcular el volumen

Volumen = (4/3) * PI * radio**3

!------------------------------------------------------------------

! Brindar la informacion

WRITE (*,*) " El volumen de la esfera es de = ", Volumen

!------------------------------------------------------------------


END PROGRAM Volumen_Esfera


\end{Verbatim}


\subsection*{Precisi\'on}
Este programa determina la precisi\'on de la m\'aquina. Se compara repetidamente $1 + \epsilon_m$ con 1 a medida que $\epsilon_m$ se vuelve mas peque\~no y se muestra como impacta en la precisi\'on del c\'alculo.\\ \\
\adjincludegraphics[scale=0.67,clip]{Precision}
	
\begin{Verbatim}[frame=single]

! Limits.f90 : Determines machines precision
!---------------------------------------------

Program Limits
Implicit none
Integer :: i , n
Real *8 :: epsilon_m , one
n=60 ! Establish the number of iterations
! Set initial values : 
epsilon_m = 1.0
one = 1.0
! Within a DO-LOOP, calculate each step and print.
! This loop will execute 60 times in a row as i is
! incremented form 1 to n ( since n = 60 ):
do i = 1, n , 1 ! Begin the do-loop
epsilon_m = epsilon_m / 2.0 ! Reduce epsilon m 
one = 1.0 + epsilon_m ! Recalculate one
print * , i , one , epsilon_m ! Print values so far
end do ! End loop when i > n 
End program Limits

\end{Verbatim}

\subsection*{Precisi\'on4}
Se modifica el programa anterior cambiado la precisi\'on a sencilla.\\ \\
\adjincludegraphics[scale=0.67,clip]{Precision4}
	
		
\begin{Verbatim}[frame=single]
! Limits.f90 : Determines machines precision
!---------------------------------------------

Program Limits
Implicit none
Integer :: i , n
Real *4 :: epsilon_m , one
n=60 ! Establish the number of iterations
! Set initial values : 
epsilon_m = 1.0
one = 1.0
! Within a DO-LOOP, calculate each step and print.
! This loop will execute 60 times in a row as i is
! incremented form 1 to n ( since n = 60 ):
do i = 1, n , 1 ! Begin the do-loop
epsilon_m = epsilon_m / 2.0 ! Reduce epsilon m 
one = 1.0 + epsilon_m ! Recalculate one
print * , i , one , epsilon_m ! Print values so far
end do ! End loop when i > n 
End program Limits

\end{Verbatim}


\subsection*{Funciones}
Se muestra el uso de algunas funciones matematicas en Fortran y se imprimen en la pantalla de terminal los ejemplos de sus valores.\\ \\
\adjincludegraphics[scale=0.67,clip]{Funciones}

\begin{Verbatim}[frame=single]

! Math . f90 : demo some Fortran math functions
!-----------------------------------------------

Program Math_test ! Begin main program
Real *8 :: x = 1.0 , y, z ! Declare variables x, y, z

y = sin (x) ! Call the sine function
z = exp (x) + 1.0 ! Call the exponential function
print * , x, y, z ! Print x, y, z
End program Math_test ! End main program

\end{Verbatim}


\subsection*{Mathtest2}
Se expone el uso de variables complejas para obtener los valores que se pide.\\ \\

	
	
		
\begin{Verbatim}[frame=single]
Program Mathtest2
  
  COMPLEX, PARAMETER    :: MINUS_ONE = -1.0, x= 2.0
  COMPLEX               :: Z,y
  Real *8 :: w, q=1.0

  Z = SQRT(MINUS_ONE)
  y=asin(x)
  w=log(q-1)

  
  Print * , Z, y, w

End Program Mathtest2

\end{Verbatim}
	
\subsection*{Funci\'onXY}
Aqu\'i se muestra el uso de funciones definidas en el c\'odigo.\\ \\
\adjincludegraphics[scale=0.67,clip]{FuncionXY}
	
	
		
\begin{Verbatim}[frame=single]
Real *8 Function f (x,y)
  Implicit None

  Real *8 :: x,y 

  f= 1.0 + sin(x*y)
End Function f

Program Main

  Implicit None
  Real *8 :: Xin=0.25, Yin=2.0 , c, f

  c=f(Xin,Yin)
  write (*,*) 'f(Xin,Yin)=' , c

End Program Main

\end{Verbatim}



\subsection*{Subrutina}
Y un ejemplo del manejo de subrutinas. En el c\'odigo, se escribe la rutina a realizar antes de iniciar el programa, la rutina se llama al momento de requerirla especificando los par\'ametros.\\ \\
\adjincludegraphics[scale=0.67,clip]{Subrutina}


	
\begin{Verbatim}[frame=single]
Subroutine g(x,y,ans1,ans2)
  Implicit None
  Real (8) :: x,y,ans1,ans2

  ans1=sin(x*y) + 1 
  ans2=ans1**2

End Subroutine g

Program Main_program

  Implicit None
  Real *8 :: Xin=0.25, Yin=2.0, Gout1, Gout2

  call g(Xin,Yin,Gout1,Gout2)
  write (*,*) 'The answers are:' , Gout1, Gout2

End Program Main_program
\end{Verbatim}


\end{document}